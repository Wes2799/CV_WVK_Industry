\begin{rSection}{Research Experience}

{\bf Doctor of Philosophy: \textit{The Effects of Small and Large Scale Structure on Galaxy Evolution in the Local Universe}}

\begin{rSubsection}{Analysis of Star Formation as a Function of Environment Using Improved Halo Mass Metrics}{September 2025 -- Present}{}{}
\item Applying advanced halo mass estimators as environmental tracers to quantify star formation suppression across 24,000+ galaxies, demonstrating superior predictive power compared to traditional proximity-based metrics
\item Developing comprehensive data visualisations translating complex multi-dimensional relationships into actionable insights for scientific researchers
\item Leveraging scientific writing and collaborative documentation (Overleaf) to communicate findings effectively for publication (Van Kempen et al. in prep.)
\end{rSubsection}

\begin{rSubsection}{Evaluation and Validation of Halo Mass Estimation Models}{June 2025 -- August 2025}{}{}
\item Conducted comparative analysis of newly developed halo mass estimators against existing methods, identifying systematic biases and quantifying performance improvements through residual analysis and error propagation
\item Created publication-quality visualisations highlighting model strengths and limitations, facilitating data-driven decision-making for survey design
\item Published findings demonstrating methodological improvements in peer-reviewed journal (Van Kempen et al. 2025)
\end{rSubsection}

\begin{rSubsection}{Bayesian Hyperparameter Optimisation for Predictive Halo Mass Relations}{October 2024 -- June 2025}{}{}
\item Collaborated with ICRAR UWA team to develop dual MCMC-based approaches (emcee, PyMC) for dark matter halo mass estimation, achieving improved accuracy and uncertainty quantification on simulated datasets
\item Performed systematic model selection using AIC and BIC criteria across multiple functional forms, identifying optimal architectures balancing complexity and predictive performance
\item Delivered calibrated mass estimation relations adopted for integration into European Southern Observatory's Wide-Area VISTA Extragalactic Survey and 4MOST Hemisphere Survey catalogues
\item Published methodology and results in peer-reviewed publication (Van Kempen et al. 2025), with relations utilised in ongoing research (Van Kempen et al. in prep.)
\end{rSubsection}

\begin{rSubsection}{Multi-Scale Environmental Analysis of Star Formation in the Local Universe}{February 2024 -- September 2024}{}{}
\item Synthesised multiple analysis pipelines to investigate environmental regulation of star formation across field, pair, and group galaxy populations
\item Designed advanced data visualisations communicating complex statistical relationships and environmental dependencies to diverse audiences
\item Demonstrated quantifiable environmental impacts on star formation through rigorous statistical analysis, published in peer-reviewed journal (Van Kempen et al. 2024)
\item Applied scientific writing best practices and collaborative tools (Overleaf) for manuscript preparation and scientific communications
\end{rSubsection}

\begin{rSubsection}{Automated Duplicate Detection and Galaxy Pair Identification Pipeline}{July 2023 -- February 2024}{}{}
\item Implemented KDTree spatial indexing algorithm for efficient nearest-neighbour searches, identifying galaxy pairs in large catalogues based on velocity and angular separation criteria
\item Developed automated quality control pipeline using astroquery and os libraries to generate comparative visualisations, flagging duplicate source measurements for manual review
\item Created end-to-end data cleaning workflow improving dataset integrity, with cleaned sample utilised in published analysis (Van Kempen et al. 2024)
\item Reduced manual inspection time by 70\% through automation whilst maintaining high accuracy in duplicate identification
\end{rSubsection}

\begin{rSubsection}{Bayesian Correction Framework for Survey Selection Effects}{May 2023 -- July 2023}{}{}
\item Designed and implemented Bayesian statistical correction models addressing magnitude-limited survey biases in group membership assignments
\item Standardised environmental classifications across heterogeneous datasets, enabling valid cross-sample comparisons and eliminating systematic offsets
\item Applied corrected membership probabilities as key environmental metric in star formation analysis, published in Van Kempen et al. (2024)
\item Ensured statistical validity of downstream analyses through rigorous uncertainty propagation and bias correction
\end{rSubsection}

\begin{rSubsection}{Friends-of-Friends Algorithm Optimisation for 3D Galaxy Groups}{September 2022 -- May 2023}{}{}
\item Optimised a friends-of-friends (FoF) graph-theory based algorithm (pyFoF package) for extracting galaxy groups from 3D spatial and redshift distributions
\item Collaborated directly with package developer via GitHub to implement bug fixes and performance enhancements, contributing to open-source scientific software
\item Calibrated algorithm parameters using simulated mock catalogues with realistic observational effects (redshift-space distortions, survey masks), validated through 3D visualisation tools (Partiview, iDaVIE VR)
\item Produced robust, well-defined group catalogues utilised across three publications: Van Kempen et al. (2024), Van Kempen et al. (2025), and Van Kempen et al. (in prep.)
\item Achieved high completeness and low contamination rates through systematic parameter optimisation and visual validation
\end{rSubsection}

%%%%%%%%%%%%%%%%%%%%%%%%%%%%%%%%%%%%%%%%%%

{\bf Bachelor of Science - Honours: \textit{Characterising Star Formation in Nearby Resolved Galaxies}}

\begin{rSubsection}{Spatially-Resolved Multi-Wavelength Star Formation Rate Indicator Validation}{July 2021 -- June 2022}{}{}
\item Developed automated image processing pipeline for multi-wavelength astronomical data, implementing spatial alignment algorithms, resolution standardisation to common pixel scales, and unit conversion workflows across UV, IR, and SED-modelled datasets
\item Performed systematic unit conversion and flux calibration across heterogeneous telescope data sources, transforming diverse measurement units into standardised star formation rate estimates (solar masses per year)
\item Conducted pixel-by-pixel comparative analysis of monochromatic star formation rate indicators across spatially-resolved regions of nearby galaxies, identifying systematic offsets and regional dependencies through correlation analysis
\item Validated reliability and physical applicability of different star formation tracers through residual diagnostics and statistical testing, quantifying measurement uncertainties and calibration differences between UV and IR methods
\item Implemented quality control frameworks ensuring data consistency and fair comparison across multi-band measurements at matched spatial resolutions
\item Honours research contributed to methodology and analysis published in co-authored peer-reviewed paper (Cluver et al. 2025)
\end{rSubsection}

\end{rSection}